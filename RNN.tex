%% Copyright 2009 Elsevier Ltd
%%
%% This file is part of the 'Elsarticle Bundle'.
%% ---------------------------------------------
%%
%% It may be distributed under the conditions of the LaTeX Project Public
%% License, either version 1.2 of this license or (at your option) any
%% later version.  The latest version of this license is in
%%    http://www.latex-project.org/lppl.txt
%% and version 1.2 or later is part of all distributions of LaTeX
%% version 1999/12/01 or later.

\documentclass[preprint,12pt]{elsarticle}
\usepackage{graphicx}
\usepackage{lineno}

%\journal{Journal Name}

\begin{document}

\begin{frontmatter}

\title{Using Recurrent Neural Nets to Identify Isolated Leptons in High Energy Particle Collision Data}

\author{Ben Hooberman, Anil Radhakrishnan, Matt Zhang}

\address{UIUC}

\begin{abstract}
We demonstrate the use of recurrent neural nets (RNNs) for classifying leptons in particle collisions as either coming from prompt interactions or heavy flavor decays. This method is shown to improve upon current methods for lepton isolation, such as ptcone and PLT.
\end{abstract}

\end{frontmatter}

\section{Introduction}\label{sec:intro}

[WHY LEPTON ISOLATION IS IMPORTANT, WHAT ANALYSES IT'S USED IN, ETC]

[EXISTING TECHNIQUES, PTCONE, PLT, ETC]

[USE OF MACHINE LEARNING IN HEP, AND WHY RNN WOULD BE USEFUL]

The codebase used in this project can be found at github.com/BucketOfFish/LeptonIsolation. Training is performed with PyTorch.

\section{Event Selection}\label{sec:selection}

The leptons used in this project, both prompt and heavy-flavor, were extracted from a Monte-Carlo ttbar sample.
%To be specific, we used the sample $mc16_13TeV.410470.PhPy8EG_A14_ttbar_hdamp258p75_nonallhad.deriv.DAOD_MUON5.e6337_e5984_s3126_r10201_r10210_p3584$.
After lepton and track selections were applied, we ended up extracting 8241 heavy flavor and 8241 isolated leptons, with their associated tracks.

Electrons were selected using the “DFCommonElectronsLHMedium” decorator, and muons were selected using MuonSelectionTool set at medium. Furthermore, we only kept leptons which had at least one associated track in the surrounding region.

[TRACK SELECTIONS]

\section{Data Preparation}\label{sec:dataprep}

[DATA FORMAT, ROOT TO H5]

For each lepton, we stored the following information: pdgID, pT, eta, phi, d0, z0, ptcone(20/30/40), ptvarcone(20/30/40), and truthType. pdgID and truthType were truth information, specifying the lepton flavor (electron vs. muon) and lepton isolation (heavy flavor vs. prompt) respectively. For tracks, we stored dR, dEta, dPhi, dd0, dz0, charge, eta, pT, theta, d0, z0, and chiSquared. dR, dEta, etc. for each track were calculated with respect to the track's associated lepton.

[FEATURE COMPARISONS]

\section{Ptcone Validation}\label{sec:ptcone}

[HOW PTCONE IS CALCULATED]

[COMPARISON PLOTS]

\section{RNN Architecture}\label{sec:architecture}

[ORDERING INVESTIGATION (PT, DR, ETA, ETC.)]

[HYPERPARAMETER OPTIMIZATION]

[RNN VS GRU VS LSTM]

[RNN VS POINT CLOUD]

\section{Results Comparison}\label{sec:results}

[PTCONE OPTIMIZATION]

[PLT OPTIMIZATION]

[ROC CURVE COMPARISON]

\section{Conclusion}\label{sec:conclusion}

\bibliographystyle{bibstyle}
\bibliography{bibliography.bib}

\end{document}